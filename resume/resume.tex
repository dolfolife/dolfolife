%%%%%%%%%%%%%%%%%
% This is a CV template created using altacv.cls
% (v1.6.4, 13 Nov 2021) written by LianTze Lim (liantze@gmail.com). Now compiles with pdfLaTeX, XeLaTeX and LuaLaTeX.
%
%% It may be distributed and/or modified under the
%% conditions of the LaTeX Project Public License, either version 1.3
%% of this license or (at your option) any later version.
%% The latest version of this license is in
%%    http://www.latex-project.org/lppl.txt
%% and version 1.3 or later is part of all distributions of LaTeX
%% version 2003/12/01 or later.
%%%%%%%%%%%%%%%%

%% If you need to pass whatever options to xcolor
\PassOptionsToPackage{dvipsnames}{xcolor}

\documentclass[10pt,a4paper,ragged2e,withhyper]{altacv}
%% AltaCV uses the fontawesome5 and packages.
%% See http://texdoc.net/pkg/fontawesome5 for full list of symbols.

% Change the page layout if you need to
\geometry{left=1.25cm,right=1.25cm,top=1.5cm,bottom=1.5cm,columnsep=1.2cm}

% The paracol package lets you typeset columns of text in parallel
\usepackage{paracol}

% Change the font if you want to, depending on whether
% you're using pdflatex or xelatex/lualatex
\ifxetexorluatex
  % If using xelatex or lualatex:
  \setmainfont{Roboto Slab}
  \setsansfont{Lato}
  \renewcommand{\familydefault}{\sfdefault}
\else
  % If using pdflatex:
  \usepackage[rm]{roboto}
  \usepackage[defaultsans]{lato}
  % \usepackage{sourcesanspro}
  \renewcommand{\familydefault}{\sfdefault}
\fi

% Change the colours if you want to
\definecolor{SlateGrey}{HTML}{2E2E2E}
\definecolor{LightGrey}{HTML}{666666}
\definecolor{DarkPastelBlue}{HTML}{1E3F66}
\definecolor{PastelBlue}{HTML}{779ECB}
\definecolor{GoldenEarth}{HTML}{E7D192}
\colorlet{name}{black}
\colorlet{tagline}{PastelBlue}
\colorlet{heading}{DarkPastelBlue}
\colorlet{headingrule}{GoldenEarth}
\colorlet{subheading}{PastelBlue}
\colorlet{accent}{PastelBlue}
\colorlet{emphasis}{SlateGrey}
\colorlet{body}{LightGrey}

% Change some fonts, if necessary
\renewcommand{\namefont}{\Huge\rmfamily\bfseries}
\renewcommand{\personalinfofont}{\footnotesize}
\renewcommand{\cvsectionfont}{\LARGE\rmfamily\bfseries}
\renewcommand{\cvsubsectionfont}{\large\bfseries}


% Change the bullets for itemize and rating marker
% for \cvskill if you want to
\renewcommand{\itemmarker}{{\small\textbullet}}
\renewcommand{\ratingmarker}{\faCircle}

%% Use (and optionally edit if necessary) this .cfg if you
%% want to use an author-year reference style like APA(6)
%% for your publication list
\usepackage[backend=biber,style=apa6,sorting=ydnt]{biblatex}
\defbibheading{pubtype}{\cvsubsection{#1}}
\renewcommand{\bibsetup}{\vspace*{-\baselineskip}}
\AtEveryBibitem{\makebox[\bibhang][l]{\itemmarker}}
\setlength{\bibitemsep}{0.25\baselineskip}
\setlength{\bibhang}{1.25em}


\begin{document}
\name{Rodolfo Sanchez}
\tagline{Software Engineering Leader}

\personalinfo{%
  \email{me@dolfo.codes}
  \homepage{dolfo.codes}
  \linkedin{dolfolife}
  \github{dolfolife}
}

\begin{fullwidth}
\makecvheader
\end{fullwidth}
%% Depending on your tastes, you may want to make fonts of itemized environments slightly smaller
% \AtBeginEnvironment{itemize}{\small}
\cvsection{Who Am I?}
\cvevent{Rodolfo is an Engineering Leader with a decade of experience leading teams and delivering data-intensive applications.}{}{}{}
\cvevent{He has focused his career on building cloud-native platforms since 2016 and with a background is on AI and ML he tries to break into the current world of modern AI/ML.}{}{}{}

%% Set the left/right column width ratio to 6:4.
\columnratio{0.5}

% Start a 2-column paracol. Both the left and right columns will automatically
% break across pages if things get too long.
\begin{paracol}{2}

\cvsection{Experience}

\cvevent{Software Engineering Manager}{VMware}{Jan 2020 -- Feb 2024}{Remote, US}
\begin{itemize}
    \item Led the Eventing Runtime and Function Buildpacks for the Tanzu Application Platform from day 0 to GA.
    \item Designed, architected, and Led the internal application of DORA metrics for the Tanzu organization.
\end{itemize}
\divider

\cvevent{Software Engineering Manager}{Pivotal}{Jan 2019 -- Dec 2019}{CA, US}
\begin{itemize}
    \item Engineering Manager of the networking components such as the ingress router and the c2c components of Cloud Foundry and Cloud Foundry on Kubernetes.
    \item Led working sessions about Kubernetes across the organization that helped built the foundation to migrate Cloud Foundry to Kubernetes.
\end{itemize}

\divider

\cvevent{Principal Engineer}{Pivotal}{May 2016 -- Dec 2018}{IL, US}
\begin{itemize}
    \item Led the development of solutions to integrate new products into existing systems and oversaw the transition of large, older applications to a more efficient microservices setup.
    \item Built teams and organizations setting the engineering practices and standard for high-functional teams.
\end{itemize}

\divider

\cvevent{Previous SE positions}{Sears Holdings, Enova, and others}{2009 -- 2016}{IL, US}
\begin{itemize}
    \item Designed, archited, and developed data-intensive products for small teams to big organizations.
\end{itemize}

\medskip


%% Switch to the right column. This will now automatically move to the second
%% page if the content is too long.
\switchcolumn

\cvsection{A Day of My Life}

\wheelchart{1.5cm}{0.5cm}{%
  6/8em/accent!40/Strategize and Plan the future of projects,
  3/8em/accent!30/Learning new technologies and management skills,
  8/8em/accent!60/{Building, Coaching, and mentoring teams},
  2/10em/accent/Coffee,
  3/10em/accent/Coding Side projects,
  2/6em/accent!20/Spending time with family
}
\cvsection{My Life Philosophy}

\begin{quote}
''Goals, metrics, and knowledge are the foundation to move forward''
\end{quote}

\cvsection{Most Proud of}

\cvachievement{\faTrophy}{Manage multiple teams}{learning to prioritize work better}

\divider

\cvachievement{\faHeartbeat}{Team alignment}{I love working on aligning teams with the organization}

\cvsection{Strengths}

\cvtag{Empathetic leadership mindset}
\cvtag{Eye for detail}\\
\cvtag{Motivator \& Leader}

\divider\smallskip

\cvtag{Golang}
\cvtag{Distributed Systems}\\
\cvtag{Kubernetes}
\cvtag{Serverless}\\
\cvtag{Streaming}
\cvtag{Event Driven Architectures}\\

\medskip

\cvsection{Education}

\cvevent{B.Sc.\ in Computer Engineering}{Major on AI and ML}{Simon Bolivar University}{2005 -- 2010}
\end{paracol}

\end{document}
